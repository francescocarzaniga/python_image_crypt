\documentclass[
parskip=full,               %ganze leere linie vor neuen absatz <false(einzug),half halbe linie abstand>
12pt,                       %schriftgrösse <10pt, 11pt, 12pt>
twoside,                    %modus einseitige seiten <twoside>
a4paper                     %a4 format
]{article}           

\usepackage[]{acronym}		% printonlyused: muss am Schluss noch eingefügt werden
\usepackage{graphicx}		%Use pdf, png, jpg, or eps§ with pdflatex; use eps in DVI mode
\usepackage[ngerman, english]{babel}	%Deutsche Silben Trennung	
\usepackage[T1]{fontenc}    %Silbentrennung	
\usepackage{float}  		 %bildposition erzwingen
\usepackage[toc,page]{appendix}   %erweiterte anhang umgebung
\usepackage{a4wide}
\usepackage[margin=10pt,font=small,labelfont=bf,format=plain]{caption}	%Package für kleinere Schrift unter den Bildern
\usepackage{apacite}
\usepackage{xcolor} % Required for specifying custom colors
\usepackage{fix-cm} % Allows increasing the font size of specific fonts beyond LaTeX default specifications
\usepackage{amsmath} %math package for formulas, etc. 
\usepackage[utf8]{inputenc}
\usepackage{pdfpages}
\setlength{\parindent}{0em}
\setlength{\parskip}{0.7em}



\begin{document}
	
	\begin{titlepage}

\newcommand{\HRule}{\rule{\linewidth}{0.5mm}} % Defines a new command for the horizontal lines, change thickness here

\center % Center everything on the page
 
%----------------------------------------------------------------------------------------
%	HEADING SECTIONS
%----------------------------------------------------------------------------------------
\begin{center}
\includegraphics[scale = 1]{./uzhlogo}
\end{center} 
\textsc{\LARGE University Of Zürich}\\[1.5cm] % Name of your university/college
\Large MAT 101 - Programming\\
\Large Group Project\\
\Large Fall semester 2016 \\[0.5cm]
\large Submission date: 14.12.2016 \\[0.5cm] % Minor heading such as course title

%----------------------------------------------------------------------------------------
%	TITLE SECTION
%----------------------------------------------------------------------------------------

\HRule \\[0.4cm]
{ \huge \bfseries Embedding Secret Messages Into Image Files }\\[0.4cm] % Title of your document
\HRule \\[1.5cm]
 
%----------------------------------------------------------------------------------------
%	AUTHOR SECTION
%----------------------------------------------------------------------------------------

\begin{minipage}{0.4\textwidth}
\centering
\emph{Group members:}\\ 
Lucas Pelloni, 13-722-038\\
Francesco Carzaniga,  \\
Sonia Donati,  \\
\end{minipage}


% If you don't want a supervisor, uncomment the two lines below and remove the section above
%\Large \emph{Author:}\\
%John \textsc{Smith}\\[3cm] % Your name

%----------------------------------------------------------------------------------------
%	DATE SECTION
%----------------------------------------------------------------------------------------
%{\large \today}\\[3cm] % Date, change the \today to a set date if you want to be precise

%----------------------------------------------------------------------------------------
%	LOGO SECTION
%----------------------------------------------------------------------------------------

%\includegraphics{Logo}\\[1cm] % Include a department/university logo - this will require the graphicx package
 
%----------------------------------------------------------------------------------------

\vfill % Fill the rest of the page with whitespace

\end{titlepage}	
	\section{Description of the project}
	The goal of our project is to write a program that allows us to embed (secret) 
	messages as invisible data into (PNG) images. In a second step we are able to 
	embed another image by a similar approach. 

	\subsection{Results}
	 As a first step we completed the template. By doing so, we could hide a string into an image.
Then we programmed the functions for the embedding of images into other images. At the beginning we did it for images using a colour depth of 1 bit. The hidden image was heavily altered by this process and the result was not good enough for us. Therefore we improved our functions in order to hide an image with more than one bit of depth. The hidden image with our functions is basically unchanged (very similar to the original), because we have access to a higher depth of colour.
At the end we have done the necessary improvements: encryption, decryption and encoding.
Respectively we have modified the graphic user interface such that the entire project would work.

	

\end{document}